\documentclass[letterpaper,10pt,onecolumn]{article}
\usepackage[spanish]{babel}
\usepackage[latin1]{inputenc}
\usepackage[pdftex]{color,graphicx}
\usepackage{hyperref}
\setlength{\oddsidemargin}{0cm}
\setlength{\textwidth}{490pt}
\setlength{\topmargin}{-40pt}
\addtolength{\hoffset}{-0.3cm}
\addtolength{\textheight}{4cm}

\begin{document}
\begin{center}

\includegraphics[width=490pt]{header.png}\\[0.5cm]

\textsc{\LARGE Astronom\'ia Popular}\\[0.1cm]

\large Jaime E. Forero Romero\\[0.5cm]

\end{center}

\large \noindent\textsc{Nombre del curso:} Astronom\'ia Popular %Aqui
                                %nombre del curso 
 
\noindent\textsc{C\'odigo del curso:} FISI-XXXX (CBU TIPO-E)%Aqui el
                                %codigo del curso 

\noindent\textsc{Unidad acad\'emica:} Departamento de F\'isica 

\noindent\textsc{Periodo acad\'emico:} 201520 %Aqui el periodo,
                                %p.ej. 201510 

\noindent\textsc{Horario:} %Aqui el horario, p.ej. Ma y Ju, 10:00 a
                           %11:20 

\noindent\rule{\textwidth}{1pt}\\[-0.3cm]

\normalsize \noindent\textsc{Nombre profesor(a) principal:} Jaime
E. Forero-Romero %Aqui nombre del profesor principal 

\noindent\textsc{Correo electr\'onico:}
\href{mailto:je.forero@uniandes.edu.co}{\nolinkurl{je.forero@uniandes.edu.co}}
%Cambie address por su direccion de correo uniandes 

\noindent\textsc{Horario y lugar de atenci\'on:} Ip208 %Aqui su
                                %horario y lugar de atencion,
                                %p.ej. Vi, 15:00 a 17:00, Oficina
                                %Ip102 
\\[-0.1cm]


\noindent\rule{\textwidth}{1pt}\\[-0.1cm]

\newcounter{mysection}
\addtocounter{mysection}{1}

\noindent\textbf{\large \Roman{mysection} \quad
  Introducci\'on}\\[-0.2cm] 

%Este espacio es para hacer una introduccion al curso, evidenciando la propuesta metodologica. Debe ser clara y precisa.

\noindent\normalsize 
Varios temas astrono\'omicos son parte de la
informaci\'on que recibimos a diario. El d\'ia y la noche, la
ciencia ficci\'on, vida en otros planetas, el origen del Universo,
posibles apocalipsis astron\'omicos y la exploraci\'on espacial son
algunas de las palabras que nos parecen familiares a todos.

\noindent\normalsize 
El objetivo principal del curso es tratar aspectos de la cultura
popular relacionados con la astronom\'ia y presentarlos a trav\'es del
conocimiento que se ha constru\'ido con m\'etodos cient\'ificos en
cosmolog\'ia, ciencias planetarias, f\'isica estelar, astronom\'ia de
posici\'on, coheter\'ia, ingenier\'ia aeroespacial y astrobiolog\'ia,
entre otros.

\noindent\normalsize 
El curso es apto para estudiantes de cualquier disciplina y nivel de
estudios. El \'unico requisito es el inter\'es por los temas a
tratar. \\[0.1cm]

\stepcounter{mysection}
\noindent\textbf{\large \Roman{mysection} \quad Objetivos}\\[-0.2cm]

%En este espacio se debe precisar el ente visor del curso y el proposito ideal al finalizar el curso.

\noindent\normalsize Los objetivos principales del curso son:

\begin{itemize}
\item Reconocer el lugar del conocimiento astron\'omico en diferentes
  aspectos de la cultura popular. \\[-0.6cm] 

\item Examinar los v\'inculos hist\'oricos de la astronom\'ia con
  otras \'areas del conocimiento.  \\[-0.6cm] 

\item Reconocer diferentes maneras de construir y validar conocimiento
  en las ciencias naturales, las artes y las ciencias sociales. \\[-0.2cm]
\end{itemize}

\stepcounter{mysection}
\noindent\textbf{\large \Roman{mysection} \quad Competencias a
  desarrollar}\\[-0.2cm] 

%En este espacio se describen las habilidades que el estudiante
%desarrollara en el transcurso del curso. 

\noindent\normalsize Al finalizar el curso, se espera que el
estudiante est\'e en capacidad de: 

\begin{itemize}
\item Analizar .\\[-0.6cm]
\item Leer .\\[-0.6cm]
\item Escribir un texto sobre aspectos de la cultura popular
  influenciados por la astronom\'ia para evidenciar la relaci\'on del
  conocimiento cient\'ifico con las creencias del p\'ublico no
  especializado.\\[-0.2cm]  
\end{itemize}

\stepcounter{mysection}
\noindent\textbf{\large \Roman{mysection} \quad Contenido por
  semanas}\\[-0.2cm] 

%Se expone de forma ordenada toda la tematica a tratar del curso. Debe
%planearse para 15 semanas. 

\noindent\normalsize \textbf{\textsc{Semana 1.}} El comienzo. Las
diferentes historias de origen en los \'ultimos dos mil a\~nos
en diferentes civilizaciones de Asia, Africa y Am\'erica. La teor
'ia del Big Bang.\\[-0.3cm]  


\noindent\textbf{\textsc{Semana 2.}} Espacio y tiempo. Modelos de
espacio y tiempo desde Grecia antigua hasta Einstein. \\[-0.3cm] 

\noindent\textbf{\textsc{Semana 3.}} Ritmos de la vida diaria. Ciclos
terrestres, lunares y solares que dictan ritmos biol\'ogicos y
estructuran sociedades. \\[-0.3cm] 

\noindent\textbf{\textsc{Semana 4.}} Ritmos de la vida
religiosa. Bases astron\'omicas presentes en la celebraci\'on de ritos
religiosos. \\[-0.3cm] 


\noindent\textbf{\textsc{Semana 5.}} Signos. Lo que la ciencia
contempor\'anea tiene que decir sobre la astrolog\'ia. \\[-0.3cm] 

\noindent\textbf{\textsc{Semana 6.}} El Espacio visto por los
pol\'iticos. La carrera espacial y su lugar en luchas de poder global
durante el Siglo XX y XXI.\\[-0.3cm]  

\noindent\textbf{\textsc{Semana 7.}} El Universo visto por los
periodistas. Los errores y aciertos de los peridiostans en
cubrimientos de noticias sobre astronom\'ia. \\[-0.3cm]   

\noindent\textbf{\textsc{Semana 8.}} Vida m\'as all\'a de la
Tierra. Astrobiolog\'ia b\'asica y la b\'usqueda de la ciencia de vida
extraterrestre. \\[-0.3cm]  

\noindent\textbf{\textsc{Semana 9.}} Nuevos mundos. Exoplanetas y la
b\'uesqueda de nuevos mundos habitables .\\[-0.3cm] 

\noindent\textbf{\textsc{Semana 10.}} Humanos en el espacio. Historias
de Cosmonautas y astronautas. La posible colonizaci\'on del sistema
solar. \\[-0.3cm] 

\noindent\textbf{\textsc{Semana 11.}} El espacio exterior en la
ciencia ficci\'on contempor\'anea. Paseo por la obra de Isaac Asimov,
Carl Sagan y Philip K. Dick.\\[-0.3cm] 

\noindent\textbf{\textsc{Semana 12.}} De pel\'icula. Referencias en el
cine de ciencia ficci\'on a la exploraci\'on espacial. \\[-0.3cm] 

\noindent\textbf{\textsc{Semana 13.}} Banda Sonora. Referencias de la
m\'usica popular a temas astron\'omicos. La ciencia detr\'as de
diferentes composiciones musicales. \\[-0.3cm] 

\noindent\textbf{\textsc{Semana 14.}} Space-art. El punto de encuentro
entre la astronom\'ia y las ciencias del espacio con las artes
pl\'asticas.\\[-0.3cm] 

\noindent\textbf{\textsc{Semana 15.}} El Fin. Historias de apocalipsis
astron\'omicos y su verdadera probabilidad de suceder.\\[0.1cm] 

\stepcounter{mysection}
\noindent\textbf{\large \Roman{mysection} \quad Metodolog\'ia}\\[-0.2cm]

%Se describen las tecnicas y metodos para el desarrollo exitoso del curso.

\noindent\normalsize Aqu\'i texto.\\[0.1cm]

\stepcounter{mysection}
\noindent\textbf{\large \Roman{mysection} \quad Criterios de evaluaci\'on}\\[-0.2cm]

%Tener en cuenta los siguientes aspectos:
%	\item Porcentajes de cada evaluacion
%	\item Fechas importantes
%	\item Parametros de calificacion
%	\item Calificacion de asistencia y/o participacion en clase
%	\item Reclamos
%	\item Politica de aproximacion de notas

\noindent\normalsize Aqu\'i texto.\\[0.1cm]

\stepcounter{mysection}
\noindent\textbf{\large \Roman{mysection} \quad Bibliograf\'ia}\\[-0.2cm]

%Indicar los libros y la documentacion guia.

\noindent\normalsize Bibliograf\'ia principal:

\begin{itemize}
	\item Autores. \textit{Titulo}, Anho. (Biblioteca General - Codigo biblio)
\end{itemize}

\noindent\normalsize Bibliograf\'ia complementaria:

\begin{itemize}
	\item Autores. \textit{Titulo}, A\~no. (Biblioteca General - Codigo biblio)\\[-0.6cm]
	\item Autores. \textit{Titulo}, A\~no. (Biblioteca General - Codigo biblio)\\[-0.6cm]
	\item Autores. \textit{Titulo}, A\~no. (Biblioteca General - Codigo biblio)
\end{itemize}

\end{document}
