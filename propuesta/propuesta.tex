\documentclass{report}
\begin{document}

\subsection*{\bf T\'itulo tentativo}
Pensamiento Gal\'actico Contempor\'aneo\\
Astronom\'ia, Arte y Misticismo\\
M\'ascaras de la Astronom\'ia\\
Reflejos Astron\'omicos\\
42\\

\subsection*{Tipo de CBU}
Tipo A. Tambi\'en se ofrece como un curso tipo E (intensivo en
escritura acad\'emica). 


\subsection*{Descripci\'on del curso}
La astronom\'ia es una ciencia que nos enfrenta con preguntas
fundamentales sobre nuestro origen y lugar en el Universo. Por
milenios ha influenciado varias actividades humanas como el arte, la
pol\'itica y la religi\'on. Este curso tratar\'a esas relaciones,
presentando el conocimiento astron\'omico moderno desde perspectivas
diferentes como el arte contempor\'aneo, culturas americanas
ancestrales, luchas de poder, religi\'on, ciencia ficci\'on y
tradiciones herm\'eticas occidentales.   
\subsection*{Objetivos}

\begin{itemize}
\item Reflexionar sobre el rol de la astronom\'ia en diferentes
  preguntas ontol\'ogicas, pol\'iticas y religiosas. 
\item Reconocer el nuevo fortalecimiento de la conexi\'on entre arte y
  astronom\'ia durante las \'ultimas d\'ecadas.
\item Identificar la influencia de la astronom\'ia en la m\'usica,
  cine y literaturas contempor\'aneas.
\end{itemize}

\subsection*{Metodolog\'ia}

Clases magistrales. Entrevistas a invitados.

\subsection*{Forma de evaluaci\'on}

Entrega de ensayos.

\subsection*{Temas}

\begin{enumerate}
\item {\bf Macrocosmos y Microcosmos}. El Universo y el orden de las
  cosas desde diferentes tradiciones herm\'eticas occidentales y
  filosof\'ias orientales.
\item {\bf Signos y Comportamiento}. Las medias verdades de la astrolog\'ia.
\item {\bf Poder}. El uso de cuerpos celestes y el espacio exterior como instrumentos de poder. 
\item {\bf Dioses Solares}. La conexi\'on astron\'omica con las grandes religiones de occidente. 
\item {\bf Conocimiento ancestral}. Arqueoastronom\'ia con \'enfasis en las culturas americanas pre-colombinas.
\item {\bf Inteligencia extraterrestre}. Las diferentes caras de
  nuestra b\'usqueda por inteligencia extraterrestre.
\item {\bf Cosmonautas}. Humanos viajando por el espacio exterior. Sus experiencias y mensajes.
\item {\bf Palabras, im\'agenes y sonidos}. Literatura, cine y
  m\'usica del \'ultimo siglo que tienen temas astron\'omics como
  referente principal.
\item {\bf Experimentos espaciales}. V\'inculos de la astronom\'ia con
  el arte contempor\'aneo.
\end{enumerate}

\subsection*{Bibliografía}

\begin{itemize}
\item de Greiff, \emph{A}, Universidad de los Andes, 1981.
\item Roob, A. \emph{Alquimia y M\'istica}, Taschen, 2005.
\item Sagan, C. \emph{The Varieties of Scientific Experience. A personal View of the Search for God}, Penguin, 2006.
\item Snow, C.P. \emph{The Two Cultures}, Cambridge, 2012.
\item Wilson, S. \emph{Information Arts. Intersections of Art, Science
and Technology}, The MIT Press, 2003.
\item Zielinski, S. y otros (Editores). \emph{Variantology - On Deep Time
  Relations Of Arts, Sciences and Technologies.} (Tomos 1-5), Verlag
der Buchhandlung Walther Koenig, 2005-2011.
\end{itemize}

\end{document}
