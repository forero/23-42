\documentclass[12pt]{report}
\begin{document}

\subsection*{\bf T\'itulo}
Astronom\'ia, Arte y Cultura\\

\subsection*{\bf Profesor}
Jaime Forero\\

\subsection*{Tipo de CBU}
Tipo E (intensivo en escritura acad\'emica). 
Como tipo A/B dependiendo de la clasificaci\'on del comit\'e de CBU.  

\subsection*{Descripci\'on del curso}
La astronom\'ia es una ciencia que nos invita a plantearnos preguntas
fundamentales sobre nuestro origen y lugar en el Universo. 
Por esta raz\'on la astronom\'ia ha alimentado el
arte, la pol\'itica, la filosof\'ia y la religi\'on. Este curso busca
hacer expl\'icitos esos v\'inculos.

Para esto se utilizar\'an resultados y discursos de diferentes
ciencias sociales como la historia, la antropolog\'ia, la arqueolog\'ia, la
sociolog\'ia y los estudios culturales. Prestaremos atenci\'on
especial a temas como las culturas americanas  ancestrales, las luchas
de poder pol\'itico, la ciencia-ficci\'on, la astrolog\'ia y la
religi\'on.  

El curso es apto para estudiantes de cualquier disciplina, el \'unico
requisito es el inter\'es por los temas a tratar en el curso. 
\subsection*{Objetivos}

\subsubsection*{Objetivo General}
\begin{itemize}
\item Reflexionar sobre el papel de la astronom\'ia dentro de la
  construcci\'on de discursos filos\'oficos,
  est\'eticos, pol\'iticos y religiosos. 
\end{itemize}

\subsubsection*{Objetivos Espec\'ificos}
\begin{itemize}
\item Mostrar a los estudiantes diferentes maneras de construir
  y validar conocimiento en las ciencias naturales, las artes y las
  ciencias sociales. 
\item Mostrar a los estudiantes los v\'inculos hist\'oricos de la
  astronom\'ia con diferentes maneras de b\'usquedas de conocimiento. 
\end{itemize}

\subsection*{Metodolog\'ia}

El curso se desarrollar\'a en su mayor\'ia a trav\'es de clases
magistrales en dos sesiones semanales de una hora y media cada
una. 
Las clases magistrales har\'an uso de transparencias, video y
audio. 

En cuatro sesiones del semestre se har\'a una conversaci\'on con
invitados especiales. 
Estas sesiones especiales est\'an listadas en la
secci\'on de Temas y Sesiones.  

En algunas sesiones (m\'aximo 3 durante el semestre) se dar\'a un
retroalimentaci\'on durante media hora sobre los ensayos escritos por los
estudiantes. 

\subsection*{Prerrequisitos}
Curiosidad sobre la relaci\'on de la astronom\'ia con las ciencias
sociales y las artes.

\subsection*{Forma de evaluaci\'on}

Entrega de cuatro ensayos durante el semestre. Cada uno de los ensayos
cuenta en un $25 \%$ de la nota final. Cada uno de los ensayos debe
tener entre 2000-2500 palabras.  

Para que los estudiantes tengan oportunidad de mejorar sus ensayos se
har\'an dos entregas. La primera servir\'a para dar una
retroalimentaci\'on inicial y tendr\'a una calificaci\'on de $10\%$
sobre la nota final. La segunda entrega ser\'a la definitiva y
tendr\'a  valor de $15\%$ sobre la nota final. 
 
Los dos primeros ensayos son textos comparativos que retoman temas
trata-dos en clase, de modo que ampl\'ien las referencias y la
discusi\'on all\'i presentados.

El tercer ensayo es de car\'acter divulgativo. Retoma un tema
tratado en clase y lo convierte en un texto que podr\'ia encontrarse
en un medio de comunicaci\'on masivo.  

El cuarto ensayo es de car\'acter argumentativo. El estudiante propone
un tema nuevo que concuerda con el programa general del curso y lo
desarrolla. El ensayo debe tomar en cuenta un aspecto
relevante de las relaciones entre arte, astronom\'ia y ritual que
no fue explorado en el curso. 

\subsection*{Temas y Sesiones}

Los temas desarrollados en el curso, desglosados clase por clase seg\'un
la programaci\'on semanal, ser\'an:


\begin{enumerate}
\item {\bf Macrocosmos y Microcosmos}
\begin{itemize}
\item[Clase 1] Introducci\'on. Motivaci\'on general del
  curso. Ejemplos de v\'inculos de arte, ciencia y ritual a lo
  largo de la historia.
\item[Clase 2] Aislando al individuo del macrocosmos. Entre alquimia, ciencia y
  misticismo. Los casos de Isaac Newton y Francis Bacon.  
\end{itemize}

\item {\bf Macrocosmos y Microcosmos}
\begin{itemize}
\item[Clase 3] El individuo como un puente entre el macrocosmos y el
  microcosmos. Los casos de las tradiciones hinduista y budista.
\item[Clase 4] El individuo como la conexi\'on entre el cielo y la
  tierra. La cosmovisi\'on de los pueblos originales de Am\'erica.
\end{itemize}

\item {\bf Signos y Comportamiento}
\begin{itemize}
\item[Clase 5] Or\'igenes de la astrolog\'ia. Las medias verdades de
  la astrolog\'ia desde la ciencia contempor\'anea. 
\item[Clase 6] {\bf Conversaci\'on con un invitado (a determinar) sobre
  la pr\'actica astrol\'ogica.} 
\end{itemize}

\item {\bf Poder}
\begin{itemize}
\item[Clase 7] En qu\'e momento lograron tener los astr\'onomos poder
  pol\'itico. Casos de Am\'erica precolombina y Europa.
\item[Clase 8] Las luchas de poder modernas en el Espacio
  Exterior. Desde la Guerra Fr\'ia hasta China en el espacio. 
  {\bf Sesi\'on de 30 minutos para la retroalimentaci\'on sobre los
    ensayos de los estudiantes}. 
\end{itemize}

\item {\bf Deidades Solares y Lunares}
\begin{itemize}
\item[Clase 9] El Sol y la Luna en los ciclos terrestres. Deidades
  Solares y Lunares en Egipto, Mexico, India y China.
\item[Clase 10] Tammuz, Osiris, Adonis y Cristo. Paralelos entre
  deidades solares.
\end{itemize}

\item {\bf Conocimiento Ancestral}
\begin{itemize}
\item[Clase 11] Introducci\'on a la arqueoastronom\'ia y
  etnoastronom\'ia. Objetivos y m\'etodos. 
\item[Clase 12] Arqueoastronom\'ia y etnoastronom\'ia con \'enfasis en
  comunidades americanas.
\end{itemize}

\item {\bf Conocimiento Ancestral}
\begin{itemize}
\item[Clase 13] Arqueoastronom\'ia y etnoastronom\'ia con \'enfasis en
  comunidades colombianas.
\item[Clase 14] Somos hijos de las estrellas. {\bf Entrevista con un
  representante de la comunidad Mhuysca de Bogot\'a.} 
\end{itemize}

\item {\bf Inteligencia Extraterrestre}
\begin{itemize}
\item[Clase 15] Astrobiolog\'ia b\'asica. Posibles definiciones de
  vida. Condiciones de habitabilidad. B\'usquedas de exoplanetas.
\item[Clase 16] Or\'igenes de vida inteligente.  B\'usqueda de
  inteligencia extraterrestre. Posibles implicaciones de encontrar
  inteligencia extraterrestre.
  {\bf Sesi\'on de 30 minutos para la retroalimentaci\'on sobre los
    ensayos de los estudiantes}. 
\end{itemize}

\item {\bf Cosmonautas}
\begin{itemize}
\item[Clase 17] Desde la carrera espacial entre Estados Unidos y la Uni\'on
  Sovi\'etica hasta \emph{Space Oddity} en la Estaci\'on Espacial
  Internacional. Las historias de Valentina, Yuri, Neil y Chris.
\item[Clase 18] \emph{The overview effect}. El cambio reportado por
  cosmonautas y astronautas despu\'es de ver la Tierra desde \'orbita.
\end{itemize}

\item {\bf Palabras, Im\'agenes y Sonidos}
\begin{itemize}
\item[Clase 19] Desde Saturno hasta Solaris. Paseo por la obra escrita
  de Voltaire, Klopstock, Scheerbart, Elfriede Jelinek y Stanis\l aw Lem.
\item[Clase 20] El espacio exterior en la ciencia-ficci\'on
  contempor\'anea. Paseo por la obra de Isaac Asimov, Carl Sagan,
  Philip K. Dick. 
\end{itemize}

\item {\bf Palabras, Im\'agenes y Sonidos}
\begin{itemize}
\item[Clase 21] El espacio exterior en el c\'omic y el cine en la
  segunda mitad del siglo XX y comienzos del siglo XX1 
\item[Clase 22] Grupos musicales con una ``conexi\'on celestial'',
  Canciones con una inspiraci\'on en la astronom\'ia.  
  {\bf Sesi\'on de retroalimentaci\'on sobre los textos de los estudiantes}.
\end{itemize}

\item {\bf Arte y Ritual}
\begin{itemize}
\item[Clase 23] La conexi\'on entre arte y ritual y su relaci\'on con
  ceremonias de ciclos astron\'omicos.
\item[Clase 24] Un ejemplo vivo de Arte y Ritual en el Teatro. La
  propuesta del Biodharma. {\bf Entrevista con Beatriz Camargo}. 
\end{itemize}

\item {\bf Experimentos Espaciales}
\begin{itemize}
\item[Clase 25] Historia de las conexiones profundas entre entre Arte
  y Astronom\'ia. Ejemplos del mundo europeo. 
\item[Clase 26]  Historia de las conexiones profundas entre entre Arte
  y Astronom\'ia. Ejemplos del mundo \'arabe y asi\'atico.
\end{itemize}

\item {\bf Experimentos Espaciales}
\begin{itemize}
\item[Clase 27] Historia de las conexiones profundas entre entre Arte
  y Astronom\'ia. Ejemplos de Latinoam\'erica. 
\item[Clase 28] Artistas pl\'asticos contempor\'aneos que toman al
  espacio y la astronom\'ia como motivaci\'on directa en su trabajo. 
\end{itemize}

\item {\bf Experimentos Espaciales}
\begin{itemize}
\item[Clase 29] Artistas colombianos con preguntas cercanas a la
  astronom\'ia. {\bf Conversaci\'on con Alejandro Tamayo}.
\item[Clase 30] El colectivo Fluid Skies. Una colaboraci\'on entre una
  historiadora, un artista y un astrof\'isico para explorar la
  materalidad fluida del Universo.
\end{itemize}

\end{enumerate}



\subsection*{Bibliograf\'ia}

\begin{itemize}

\item Arias de Greiff, J., \emph{Etnoastronom\'ias Americanas, Memorias
del 45 Congreso de Americanistas, Universidad de los Andes, Bogot\'a},
  Centro Editorial Universidad Nacional de Colombia, 1987.

\item Cosgrove, D., \emph{Apollo's Eye: A Cartographic Genealogy of
  the Earth in the Western Imagination}, Johns Hopkins University
  Press, 2003.
\item Paz, O., \emph{La llama doble: amor y erotismo}, Seix Barral, 2012.
\item Roob, A., \emph{Alquimia y M\'istica}, Taschen, 2005.
\item Sagan, C., \emph{The Varieties of Scientific Experience. A personal View of the Search for God}, Penguin, 2006.
\item Snow, C.P., \emph{The Two Cultures}, Cambridge, 2012.
\item Wilson, S., \emph{Information Arts. Intersections of Art, Science
and Technology}, The MIT Press, 2003.
\item Zielinski, S. y otros (Editores), \emph{Variantology - On Deep Time
  Relations Of Arts, Sciences and Technologies.} (Tomos 1-5), Verlag
der Buchhandlung Walther Koenig, 2005-2011.

\item Willimberg, H. P., \emph{Marsmenschen:  Wie die Au\ss erirdischen
  gesucht und erfunden werden}, Reclam Leipzig, 1997

\end{itemize}

\end{document}

\item Philip K Dick's Ubik
