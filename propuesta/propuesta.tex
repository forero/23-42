\documentclass{report}
\begin{document}

\subsection*{\bf T\'itulo}
Astronom\'ia, Arte y Ritual\\
Astronom\'ia, Arte y Poder\\

\subsection*{\bf Profesor}
Jaime Forero\\

\subsection*{Tipo de CBU}
Tipo E (intensivo en escritura acad\'emica). Secundariamente como tipo
A/B dependiendo de la clasificaci\'on del comite de CBU.  

\subsection*{Descripci\'on del curso}
La astronom\'ia es una ciencia que nos enfrenta con preguntas
fundamentales sobre nuestro origen y lugar en el Universo. Por
milenios ha influenciado varias actividades humanas como el arte, la
pol\'itica y la religi\'on. Este curso tratar\'a esas relaciones,
presentando el conocimiento astron\'omico moderno desde varias
perspectivas como el arte contempor\'aneo, culturas americanas
ancestrales, luchas de poder pol\'itico, ciencia ficci\'on,
astrolog\'ia y religi\'on. El curso es introductorio; no requiere
conocimientos previos y puede ser tomado por estudiantes de cualquier
disciplina. 
\subsection*{Objetivos}

\subsubsection*{Objetivo General}
\begin{itemize}
\item Reflexionar sobre el rol de la astronom\'ia en el marco de 
  diferentes preguntas filos\'oficas, est\'eticas, pol\'iticas y
  religiosas.
\end{itemize}

\subsubsection*{Objetivos Espec\'ificos}
\begin{itemize}
\item Mostrar a los estudiantes diferentes maneras de construir
  y validar conocimiento en las ciencias naturales, las artes y las
  ciencias humanas. 
\item Mostrar a los estudiantes los v\'inculos hist\'oricos de la
  astronom\'ia con diferentes b\'usquedas est\'eticas y espirituales.  
\end{itemize}

\subsection*{Metodolog\'ia}

El curso se desarrollar\'a en su mayor\'ia a trav\'es de clases
magistrales en dos sesiones semanales de una hora y media cada
una. Las clases magistrales har\'an uso de transparencias, video y
audio. 

En tres sesiones del semestre se har\'a una conversaci\'on con
invitados especiales. Estas sesiones especiales est\'an listadas en la
secci\'on de Temas y Sesiones.  

En algunas sesiones (m\'aximo 4 durante el semestre) se dar\'a un
retroalimentaci\'on durante media hora sobre los ensayos escritos por los
estudiantes. 

\subsection*{Prerrequisitos}
Inter\'es en las relaciones entre astronom\'ia, arte y ritual

\subsection*{Forma de evaluaci\'on}

Entrega de cuatro ensayos durante el semestre. Cada uno de los ensayos
cuenta en un $25 \%$ de la nota final. Cada uno de los ensayos debe
tener entre 2000-2500 palabras.  

Para que los estudiantes tengan oportunidad de mejorar sus ensayos se
har\'an dos entregas. La primera servir\'a para dar una
retroalimentaci\'on inicial y tendr\'a una calificaci\'on de $10\%$
sobre la nota final. La segunda entrega ser\'a la definitiva y
tendr\'a  valor de $15\%$ sobre la nota final. 
 
Los dos primeros ensayos son textos comparativos que retoman  tratados
en clase y ampl\'ian la base de las referencias y la discusi\'on dada
en clase. 

El tercer ensayo es de caracter divulgativo que retoma un tema
tratado en clase y lo convierte en un texto que podr\'ia encontrarse
en un medio de comunicaci\'on masivo.  

El cuarto ensayo es de caracter argumentativo. El estudiante propone
un tema nuevo que concuerda con el programa general del curso y lo
desarrolla. El objetivo es mostrar que se toma en cuenta un aspecto
relevante de las relaciones entre arte, astronom\'ia y ritual que
no es explorado en el curso.

\subsection*{Temas y Sesiones}

A continuaci\'on los temas presentados en una programaci\'on semanal,
clase por clase.

\begin{enumerate}
\item {\bf Macrocosmos y Microcosmos}
\begin{itemize}
\item[Clase 1] Introducci\'on. Motivaci\'on general del
  curso. Ejemplos de v\'inculos de arte, ciencia y ritual a lo
  largo de la historia.
\item[Clase 2] Alquimia.
\end{itemize}

\item {\bf Macrocosmos y Microcosmos}
\begin{itemize}
\item[Clase 3]  Los nuevos   m\'etodos. Entre alquimia, ciencia y
  misticismo.  Los casos de Isaac Newton y Francis Bacon. 
\item[Clase 4]
\end{itemize}

\item {\bf Signos y Comportamiento}
\begin{itemize}
\item[Clase 5] Or\'igenes de la astrolog\'ia. Las medias verdades de
  la astrolog\'ia desde la ciencia contempor\'anea. 
\item[Clase 6] {\bf Conversaci\'on con XXX sobre la pr\'actica
  astrol\'ogica.} 
\end{itemize}

\item {\bf Poder}
\begin{itemize}
\item[Clase 7] En qu\'e momento lograron tener los astr\'onomos poder
  pol\'itico. Casos de Am\'erica precolombina y Europa.
\item[Clase 8] Las luchas de poder modernas en el Espacio
  Exterior. Desde la Guerra Fr\'ia hasta China en el espacio.
\end{itemize}

\item {\bf Deidades Solares y Lunares}
\begin{itemize}
\item[Clase 9] El Sol y la Luna en los ciclos terrestres. Deidades
  Solares y Lunares en Egipto, Mexico, India y China.
\item[Clase 10] Tammuz, Osiris, Adonis y Cristo. Paralelos entre
  deidades solares.
\end{itemize}

\item {\bf Conocimiento Ancestral}
\begin{itemize}
\item[Clase 11]
\item[Clase 12]
\end{itemize}

\item {\bf Conocimiento Ancestral}
\begin{itemize}
\item[Clase 13]
\item[Clase 14] Comunidad Mhuysca. {\bf Entrevista con un representante de la
  comunidad Mhuysca de Bogot\'a.}
\end{itemize}

\item {\bf Inteligencia Extraterrestre}
\begin{itemize}
\item[Clase 15] Astrobiolog\'ia b\'asica. Posibles definiciones de
  vida. Condiciones de habitabilidad.
\item[Clase 16] B\'usqueda de exoplanetas. Or\'igenes de vida
  inteligente.  B\'usqueda de inteligencia extraterrestre.
\end{itemize}

\item {\bf Cosmonautas}
\begin{itemize}
\item[Clase 17] Desde la carrera espacial entre Estados Unidos y la Uni\'on
  Sovi\'etica hasta \emph{Space Oddity} en la Estaci\'on Espacial
  Internacional. Las historias de Valentina, Yuri, Neil y Chris.
\item[Clase 18] \emph{The overview effect}. El cambio reportado por
  cosmonautas y astronautas despu\'es de ver la Tierra desde \'orbita.
\end{itemize}

\item {\bf Palabras, Im\'agenes y Sonidos}
\begin{itemize}
\item[Clase 19] Desde Saturno hasta Solaris. Paseo por la obra escrita
  de Voltaire, Klopstock, Scheerbart, Elfriede Jelinek y Stanis\l aw Lem.
\item[Clase 20] El espacio exterior en la ciencia ficci\'on
  contempor\'anea. Paseo por la obra de Isaac Asimov, Carl Sagan, Philip K. Dick.
\end{itemize}

\item {\bf Palabras, Im\'agenes y Sonidos}
\begin{itemize}
\item[Clase 21] El espacio exterior en el c\'omic y el cine en la
  segunda mitad del siglo XX. 
\item[Clase 22] El espacio exterior en el c\'omic y el cine de
  comienzos del siglo XXI
\end{itemize}

\item {\bf Palabras, Im\'agenes y Sonidos}
\begin{itemize}
\item[Clase 23] Grupos musicales con una ``conexi\'on celestial''.
\item[Clase 24] Canciones con una inspiraci\'on en la astronom\'ia. 
\end{itemize}

\item {\bf Experimentos Espaciales}
\begin{itemize}
\item[Clase 25] Historia de las conexiones profundas entre entre Arte
  y Ciencia. Ejemplos del mundo \'arabe y asi\'atico.
\item[Clase 26] Historia de las conexiones profundas entre entre Arte
  y Ciencia. Ejemplos del mundo europeo. 
\end{itemize}

\item {\bf Experimentos Espaciales}
\begin{itemize}
\item[Clase 27] Artistas pl\'asticos contempor\'aneos que toman al espacio como
  motivaci\'on directa en su trabajo.
\item[Clase 28] Olafur Eliasson y el Institut f\"ur Raumexperimente.
\end{itemize}

\item {\bf Experimentos Espaciales}
\begin{itemize}
\item[Clase 29] Artistas colombianos con preguntas cercanas a la
  astronom\'ia. {\bf Conversaci\'on con Alejandro Tamayo}.
\item[Clase 30] El colectivo Fluid Skies. Una colaboraci\'on entre una
  historiadora, un artista y un astrof\'isico para explorar la
  materalidad fluida del Universo.
\end{itemize}

\end{enumerate}



\subsection*{Bibliografía}

\begin{itemize}

\item Arias de Greiff, J.\emph{Etnoastronom\'ias Americanas},
  Universidad de los Andes, 1981.

\item Cosgrove, D., \emph{Apollo's Eye: A Cartographic Genealogy of
  the Earth in the Western Imagination}, Johns Hopkins University
  Press, 2003.
\item Paz, O. \emph{La llama doble: amor y erotismo}, Seix Barral, 2012.
\item Roob, A. \emph{Alquimia y M\'istica}, Taschen, 2005.
\item Sagan, C. \emph{The Varieties of Scientific Experience. A personal View of the Search for God}, Penguin, 2006.
\item Snow, C.P. \emph{The Two Cultures}, Cambridge, 2012.
\item Wilson, S. \emph{Information Arts. Intersections of Art, Science
and Technology}, The MIT Press, 2003.
\item Zielinski, S. y otros (Editores). \emph{Variantology - On Deep Time
  Relations Of Arts, Sciences and Technologies.} (Tomos 1-5), Verlag
der Buchhandlung Walther Koenig, 2005-2011.

\item Willimberg, H. P.\emph{Marsmenschen:  Wie die Au\ss erirdischen
  gesucht und erfunden werden}, Reclam Leipzig, 1997

\end{itemize}

\end{document}

\item Philip K Dick's Ubik
